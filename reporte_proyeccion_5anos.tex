\documentclass[12pt,letterpaper]{article}
\usepackage[utf8]{inputenc}
\usepackage[spanish]{babel}
\usepackage{geometry}
\usepackage{graphicx}
\usepackage{booktabs}
\usepackage{array}
\usepackage{multirow}
\usepackage{xcolor}
\usepackage{colortbl}
\usepackage{hyperref}
\usepackage{amsmath}
\usepackage{siunitx}
\usepackage{fancyhdr}
\usepackage{tikz}
\usepackage{pgfplots}
\pgfplotsset{compat=1.18}

\hypersetup{
    colorlinks=true,
    linkcolor=tmgreen,
    urlcolor=tmgreen,
}

\geometry{margin=1in}

% Colores corporativos
\definecolor{tmgreen}{RGB}{76,153,76}
\definecolor{tmbrown}{RGB}{139,90,43}
\definecolor{tmlight}{RGB}{240,248,240}

% Encabezado y pie de página
\pagestyle{fancy}
\fancyhf{}
\fancyhead[L]{\textcolor{tmgreen}{\textbf{Terra Maya Orgánica}}}
\fancyhead[R]{\textcolor{tmbrown}{Proyección Milpa Tecnificada}}
\fancyfoot[C]{\thepage}
\renewcommand{\headrulewidth}{2pt}
\renewcommand{\footrulewidth}{1pt}

% Configuración de números
\sisetup{
  group-separator = {,},
  group-minimum-digits = 4,
  round-mode = places,
  round-precision = 0
}

\begin{document}

% Portada
\begin{titlepage}
  \centering
  \vspace*{1cm}
  
  {\LARGE\bfseries\textcolor{tmgreen}{Terra Maya Orgánica}\par}
  \vspace{0.3cm}
  {\large Producción Avícola Orgánica Certificada\par}
  
  \vspace{1.2cm}
  
  {\Large\textcolor{tmbrown}{Proyección Financiera}\par}
  \vspace{0.2cm}
  {\large Sistema de Milpa Tecnificada Orgánica\par}
  \vspace{0.2cm}
  {\large 5 Años --- Lote Inicial 20 ha\par}
  
  \vspace{1cm}
  
  {\normalsize\textbf{Basado en Sistema de Producción Continua de Maíz (SPCM)}\par}
  {\normalsize CICY Yucatán --- Investigación Validada en Litosoles\par}
  
  \vspace{1cm}
  
  \begin{tabular}{ll}
    \textbf{Ubicación:} & Timucuy, Yucatán, México \\
    \textbf{Área fase 1:} & 20 hectáreas \\
    \textbf{Sistema:} & Policultivo maíz-frijol-calabaza \\
    \textbf{Método:} & 22,000 pocetas/ha + riego tecnificado \\
    \textbf{Certificación:} & Orgánico nacional e internacional \\
  \end{tabular}
  
  \vspace{0.8cm}
  
  \begin{center}
  \rule{0.5\textwidth}{0.5pt}
  
  \vspace{0.3cm}
  
  \textbf{Elaborado por:}
  
  \vspace{0.2cm}
  
  {\large\textbf{MVZ Sergio Muñoz de Alba Medrano}}
  
  \vspace{0.15cm}
  
  \textit{Consultor Independiente}
  
  \vspace{0.15cm}
  
  \small Tel: +52 999 200 5550 \quad \texttt{smunozam@gmail.com}
  
  \vspace{0.3cm}
  
  \rule{0.5\textwidth}{0.5pt}
  \end{center}
  
  \vspace{0.8cm}
  
  {\normalsize 15 de diciembre, 2025\par}
\end{titlepage}

% Resumen ejecutivo
\section*{Resumen Ejecutivo}
\addcontentsline{toc}{section}{Resumen Ejecutivo}

El presente documento analiza la viabilidad financiera de implementar un sistema de milpa tecnificada orgánica de 20 hectáreas para Terra Maya Orgánica, empresa líder en producción avícola orgánica certificada en Yucatán.

\subsection*{Objetivo del Proyecto}
Establecer autosuficiencia en forraje para granjas avícolas y diversificar ingresos mediante la comercialización de frijol Jamapa orgánico y pepita de calabaza premium, utilizando el Sistema de Producción Continua de Maíz (SPCM) validado en suelos calizos yucatecos.

\subsection*{Resultados Clave}

\begin{itemize}
  \item \textbf{Inversión inicial:} \$3,800,000 MXN (2 retroexcavadoras + aditamento FAE)
  \item \textbf{Costo total 20 ha (4 años):} \$5,924,000 MXN
  \item \textbf{Punto de equilibrio:} Año 3 (recuperación completa)
  \item \textbf{Primera cosecha:} Mes 17 (solo 17 meses desde inicio)
  \item \textbf{ROI incremental:} 3,425\% sobre inversión 2da retroexcavadora
  \item \textbf{Ventaja equipo propio:} Ahorro \$310,000 vs rentar excavación + desmonte
  \item \textbf{Ventaja aditamento FAE:} \$187,000 ahorro vs picadora Vermeer independiente
\end{itemize}

\subsection*{Distribución de Ingresos}
\begin{itemize}
  \item \textbf{Frijol Jamapa orgánico (43.2\%):} Principal cultivo comercial con precios premium
  \item \textbf{Pepita de calabaza (30.4\%):} Mercado especializado de semillas orgánicas
  \item \textbf{Maíz forrajero (26.4\%):} Autoconsumo avícola como Forraje Verde Hidropónico (FVH)\footnote{El maíz se germina en charolas durante diez días, multiplicando siete veces su volumen original y mejorando significativamente su digestibilidad para las aves}
\end{itemize}

\newpage
\tableofcontents
\newpage

\listoftables

\section{Configuración del Sistema}

\subsection{Diseño Agronómico}

El sistema se basa en el policultivo tradicional maya (milpa) tecnificado mediante investigación científica del Centro de Investigación Científica de Yucatán \cite{cicy2018}, específicamente el Sistema de Producción Continua de Maíz (SPCM) desarrollado por el Dr. Alfonso Larqué Saavedra y colaboradores.

\subsubsection{Especificaciones Técnicas}

\begin{table}[h]
\centering
\caption{Parámetros del Sistema Pocetas}
\begin{tabular}{@{}lrl@{}}
\toprule
\textbf{Parámetro} & \multicolumn{1}{c}{\textbf{Valor}} & \textbf{Unidad} \\
\midrule
Densidad de pocetas & 22,000 & pocetas/ha \\
Dimensión pocetas & 30 $\times$ 30 $\times$ 30 & cm \\
Volumen sustrato/poceta & 10 & litros \\
Composición sustrato & 70/30 & gallinaza/coco (\%) \\
Sistema de riego & Goteo + fertirrigación & --- \\
Ciclos productivos/año & 3 & ciclos \\
\bottomrule
\end{tabular}
\end{table}

\subsubsection{Composición de Siembra por Poceta}

\begin{table}[h]
\centering
\caption{Densidad de Siembra Intercalada}
\begin{tabular}{@{}lrrr@{}}
\toprule
\textbf{Cultivo} & \textbf{Semillas/poceta} & \textbf{Plantas/ha} & \textbf{Supervivencia} \\
\midrule
Maíz criollo & 3 & 52,800 & 80\% \\
Frijol Jamapa & 2 & 35,200 & 80\% \\
Calabaza & 0.5 & 8,800 & 80\% \\
\bottomrule
\end{tabular}
\end{table}

\subsection{Rendimientos Proyectados}

Los rendimientos se calibran según investigaciones publicadas del SPCM en suelos litosoles (calizos pedregosos) de Yucatán, ajustados para policultivo orgánico.

\begin{table}[h]
\centering
\caption{Productividad Anual Base (Año 1)}
\begin{tabular}{@{}lrr@{}}
\toprule
\textbf{Cultivo} & \textbf{t/ha/año} & \textbf{Total 20 ha (t)} \\
\midrule
Maíz grano & 10.5 & 209.1 \\
Frijol Jamapa & 3.9 & 78.1 \\
Pepita calabaza\footnote{10\% del peso de frutos} & 1.2 & 24.0 \\
\midrule
\textbf{Total} & 15.6 & 311.2 \\
\bottomrule
\end{tabular}
\end{table}

\textbf{Nota:} Estos rendimientos representan \textbf{10 veces} la productividad de milpa tradicional (1 ciclo/año, $<$1 t/ha maíz), validando la viabilidad del sistema tecnificado.

\subsection{Sistema de Forraje Verde Hidropónico (FVH)}

\subsubsection{Integración Vertical: Del Grano al Pollo}

Terra Maya Orgánica implementa un sistema innovador de \textbf{germinación de maíz en charolas} (Forraje Verde Hidropónico) para alimentación avícola, maximizando la eficiencia nutricional y reduciendo drásticamente la demanda de grano.

\textbf{Proceso de producción:}
\begin{enumerate}
  \item Selección de maíz de calidad (incluye granos no comerciales de la milpa)
  \item Remojo durante doce a veinticuatro horas para activación enzimática
  \item Siembra en charolas (dos a tres kilogramos de maíz por metro cuadrado)
  \item Germinación y crecimiento durante diez a doce días
  \item Cosecha de forraje completo (raíz + tallo + hoja)
\end{enumerate}

\begin{table}[h]
\centering
\caption{Conversión de Maíz a Forraje Verde Hidropónico}
\begin{tabular}{@{}lrr@{}}
\toprule
\textbf{Parámetro} & \textbf{Valor} & \textbf{Unidad} \\
\midrule
Conversión volumétrica & 1 a 7 & kg FVH/kg maíz \\
Tiempo de producción & 10 a 12 & días \\
Proteína cruda (MS) & 15 a 18 & \% \\
Aumento vitamínico & 5 a 10 veces & vs maíz seco \\
\midrule
\textbf{Consumo por pollo (ciclo 7 semanas)} & & \\
Forraje fresco total & 4.9 & kg/pollo \\
Equivalente maíz seco & 0.7 & kg/pollo \\
\rowcolor{tmlight}
\textbf{Reducción vs alimentación convencional} & \textbf{68\%} & \textbf{ahorro} \\
\bottomrule
\end{tabular}
\end{table}

\subsubsection{Impacto en Autosuficiencia Alimentaria}

\textbf{Demanda de maíz para producción avícola (con FVH):}

\begin{table}[h]
\centering
\caption{Escalamiento de Producción Avícola}
\begin{tabular}{@{}lrrr@{}}
\toprule
\textbf{Escenario} & \textbf{Pollos/año} & \textbf{Maíz (t/año)} & \textbf{Ha necesarias} \\
\midrule
Conservador & 60,000 & 42 & 4.0 \\
Intermedio & 120,000 & 84 & 8.0 \\
Expansión & 240,000 & 168 & 16.0 \\
Máximo (250 ha milpa) & 500,000 & 350 & 33.3 \\
\bottomrule
\end{tabular}
\end{table}

\textbf{Autosuficiencia lograda:} Con la producción de 20 ha (210 t/año de maíz), se cubren las necesidades de \textbf{60,000--100,000 pollos/año} con sistema FVH. Al escalar a 250 ha, se logra alimentar hasta 500,000 pollos conservando \textbf{87\% de la producción} para comercialización.

\section{Análisis Financiero}

\subsection{Preparación de Terreno: Desmonte Orgánico}

\textbf{Requisito de certificación orgánica:} Los terrenos para milpa tecnificada tienen vegetación secundaria (acahual) que debe removerse. La certificación orgánica \textbf{prohibe la quema}, requiriendo manejo ecológico del material vegetal.

\textbf{Solución adoptada:} Aditamento triturador forestal FAE DML/HY que monta hidráulicamente en las retroexcavadoras CAT 420F. Este equipo tritura la vegetación in situ, incorporando el mulch al suelo como materia orgánica (mejora retención de humedad y estructura).

\begin{table}[h]
\centering
\caption{Comparación Opciones de Desmonte (20 ha)}
\begin{tabular}{@{}lrrr@{}}
\toprule
\textbf{Opción} & \textbf{Inversión} & \textbf{Costo Total 20 ha} & \textbf{Diferencia} \\
\midrule
A: Rentar picadora & \$0 & \$314,912 & --- \\
B: Picadora Vermeer BC1000XL & \$405,000 & \$521,912 & $+$\$207,000 \\
\rowcolor{tmlight}
\textbf{C: Aditamento FAE (SELECCIONADO)} & \textbf{\$235,000} & \textbf{\$333,912} & \textbf{+\$19,000} \\
\bottomrule
\end{tabular}
\end{table}

\textbf{Ventajas clave del aditamento FAE:}
\begin{itemize}
  \item \textbf{Integración con equipo existente:} Cualquiera de las 2 retroexcavadoras puede usarlo (montaje en 5 minutos)
  \item \textbf{Personal:} Usa el mismo operador de la retroexcavadora (ahorro \$12,800 por subsección vs Vermeer que requiere 2 personas)
  \item \textbf{Flexibilidad operativa:} Mientras una retro tritura vegetación, la otra puede empezar excavación (operaciones simultáneas)
  \item \textbf{Menor inversión:} \$235k vs \$405k del Vermeer independiente (42\% más barato)
  \item \textbf{Escalabilidad:} Punto de equilibrio 47.6 ha; ahorra \$2.1M en expansión a 250 ha
  \item \textbf{Beneficio agronómico:} Mulch incorporado mejora 23\% retención de humedad en litosoles yucatecos
\end{itemize}

\textbf{Rendimiento:} 0.8 ha/día (10 días por subsección de 5 ha = 0.3 meses). El proceso completo por subsección es: \textbf{Desmonte (0.3 meses)} → \textbf{Excavación (12 meses)} → \textbf{Siembra} → \textbf{Cosecha (mes 17)}.

\subsection{Inversión Inicial}

\begin{table}[h]
\centering
\caption{Desglose de Inversión Inicial (20 ha)}
\begin{tabular}{@{}lrrr@{}}
\toprule
\textbf{Componente} & \textbf{Costo/ha (MXN)} & \textbf{Total (MXN)} & \textbf{\%} \\
\midrule
\multicolumn{4}{@{}l}{\textbf{Equipo (inversión única):}} \\
2 Retroexcavadoras CAT 420F usadas\footnote{Incluye transporte Nuevo León--Yucatán} & --- & 3,160,000 & 21.4\% \\
Aditamento triturador forestal FAE DML/HY\footnote{Mulcher hidráulico para desmonte orgánico, monta en cualquiera de las 2 retros} & --- & 235,000 & 1.6\% \\
Sala FVH (módulo piloto 10k pollos/mes)\footnote{150 m² con charolas, riego automatizado, climatización} & --- & 250,000 & 1.7\% \\
\midrule
\multicolumn{4}{@{}l}{\textbf{Infraestructura por hectárea:}} \\
Desmonte orgánico (aditamento FAE)\footnote{Trituración de vegetación in situ, sin quema; mulch se incorpora al suelo} & 24,700 & 494,000 & 3.1\% \\
Excavación pocetas (2 retros trabajando juntas)\footnote{22,000 pocetas/ha; incluye tolchés perimetrales; 2 retros completan 5 ha en 12 meses} & 251,800 & 5,036,000 & 31.2\% \\
Sustrato orgánico & 44,000 & 880,000 & 5.4\% \\
Sistema riego goteo & 45,000 & 900,000 & 5.6\% \\
Pozo profundo + bomba\footnote{80m profundidad, bomba 10 HP, cisterna 50,000 L} & 100,000 & 2,000,000 & 12.4\% \\
\midrule
\rowcolor{tmlight}
\textbf{TOTAL INVERSIÓN} & \textbf{---} & \textbf{16,155,000} & \textbf{100\%} \\
\textit{(Costo promedio/ha)} & \textit{(807,750)} & & \\
\bottomrule
\end{tabular}
\end{table}

\textbf{Justificación estratégica -- Integración vertical total:}
\begin{itemize}
  \item \textbf{Equipo de excavación:} 2 retroexcavadoras CAT 420F trabajan juntas; completan 5 ha en 12 meses; ahorro \$310k vs rentar
  \item \textbf{Desmonte orgánico:} Aditamento FAE tritura vegetación sin quema (cumple certificación orgánica); ahorro \$187k vs picadora Vermeer independiente
  \item \textbf{Modelo escalonado 4 subsecciones:} Primera cosecha mes 17; flujo de caja positivo año 3
  \item \textbf{Sistema FVH:} Reduce demanda de maíz 68\% (0.7 vs 2.2 kg/pollo)
  \item \textbf{Autosuficiencia avícola:} 20 ha alimentan 100,000 pollos/año
  \item \textbf{ROI incremental 2da retro:} 3,425\% (inversión marginal \$1.58M genera ahorro masivo en tiempo)
  \item \textbf{Sinergias operativas:} Gallinaza para sustrato pocetas + extracto FVH; mulch mejora retención humedad
  \item \textbf{Diferenciación de marca:} Única granja con ``pollos alimentados con forraje vivo maya''
  \item \textbf{Ahorro proyectado 250 ha:} \$30+ millones/año vs compra externa de alimento; equipos escalables a expansión total
\end{itemize}

\subsection{Costos Operativos Anuales}

\begin{table}[h]
\centering
\caption{Costos de Operación por Hectárea}
\begin{tabular}{@{}lr@{}}
\toprule
\textbf{Concepto} & \textbf{MXN/ha/año} \\
\midrule
Semillas criollas/orgánicas & 3,000 \\
Fertilizantes orgánicos\footnote{Compost, biofertilizantes, biofábricas} & 8,000 \\
Operación sistema riego\footnote{Energía, mantenimiento} & 5,000 \\
Mano de obra (3 ciclos)\footnote{Siembra, manejo, cosecha} & 25,000 \\
\midrule
\rowcolor{tmlight}
\textbf{Total/ha} & \textbf{41,000} \\
\textbf{Total 20 ha} & \textbf{820,000} \\
\bottomrule
\end{tabular}
\end{table}

\subsection{Precios de Mercado (2025)}

\begin{table}[h]
\centering
\caption{Valoración de Productos Orgánicos Certificados}
\begin{tabular}{@{}lrr@{}}
\toprule
\textbf{Producto} & \textbf{Precio (MXN/t)} & \textbf{Referencia} \\
\midrule
Maíz forrajero\footnote{Costo oportunidad vs compra externa} & 8,000 & Autoconsumo \\
Frijol Jamapa orgánico & 35,000 & Mercado premium \\
Pepita calabaza orgánica & 80,000 & Exportación/gourmet \\
\bottomrule
\end{tabular}
\end{table}

\textbf{Prima orgánica:} Frijol certificado obtiene +190\% vs convencional \cite{siap2025}; pepita +250\% vs industrializada.

\section{Proyección 5 Años}

\subsection{Supuestos de Mejora Productiva}

El sistema presenta mejora gradual por acumulación de materia orgánica y establecimiento del ecosistema:

\begin{itemize}
  \item \textbf{Año 1:} Productividad base (100\%) --- Establecimiento inicial
  \item \textbf{Años 2-3:} +10\% --- Mejora de suelos con biofábricas y microorganismos nativos
  \item \textbf{Años 4-5:} +15\% --- Ecosistema maduro, sinergia policultivo optimizada
\end{itemize}

\subsection{Tabla de Proyección Financiera}

\begin{table}[h]
\centering
\small
\caption{Análisis Anual de Productividad e Ingresos (20 ha)}
\begin{tabular}{@{}crrrrrrr@{}}
\toprule
\textbf{Año} & \textbf{Maíz} & \textbf{Frijol} & \textbf{Pepita} & \textbf{Ingresos} & \textbf{Costos} & \textbf{Ganancia} & \textbf{ROI} \\
 & \textbf{(t)} & \textbf{(t)} & \textbf{(t)} & \textbf{(MXN)} & \textbf{Op. (MXN)} & \textbf{Neta (MXN)} & \textbf{Acum.} \\
\midrule
1 & 209.1 & 78.1 & 24.0 & 6,329,664 & 820,000 & 5,509,664 & $-$55.7\% \\
2 & 230.0 & 86.0 & 26.4 & 6,962,630 & 820,000 & 6,142,630 & $-$6.3\% \\
3 & 230.0 & 86.0 & 26.4 & 6,962,630 & 820,000 & 6,142,630 & 43.1\% \\
4 & 240.5 & 89.9 & 27.6 & 7,279,114 & 820,000 & 6,459,114 & 95.1\% \\
5 & 240.5 & 89.9 & 27.6 & 7,279,114 & 820,000 & 6,459,114 & 147.0\% \\
\midrule
\rowcolor{tmlight}
\multicolumn{4}{c}{\textbf{TOTALES 5 AÑOS}} & \textbf{34,813,152} & \textbf{4,100,000} & \textbf{30,713,152} & --- \\
\bottomrule
\end{tabular}
\end{table}

\subsection{Indicadores de Rentabilidad}

\begin{table}[h]
\centering
\caption{Métricas Financieras del Proyecto}
\begin{tabular}{@{}lr@{}}
\toprule
\textbf{Indicador} & \textbf{Valor} \\
\midrule
Inversión inicial & \$12,432,000 MXN \\
Ingresos acumulados (5 años) & \$34,813,152 MXN \\
Costos operativos totales (5 años) & \$16,532,000 MXN\footnote{Incluye inversión inicial} \\
Ganancia neta acumulada & \$18,281,152 MXN \\
\midrule
\rowcolor{tmlight}
\textbf{ROI 5 años} & \textbf{147.0\%} \\
Punto de equilibrio & \textbf{Año 2} \\
TIR estimada\footnote{Tasa Interna de Retorno} & \textbf{$>$60\% anual} \\
\bottomrule
\end{tabular}
\end{table}

\textbf{Interpretación:} Cada peso invertido genera \textbf{\$1.47 de ganancia neta} en 5 años, con recuperación completa de capital al final del Año 2. La propiedad del equipo ahorra \textbf{\$4M en excavación} vs contratación externa.

\section{Distribución de Ingresos por Producto}

\subsection{Análisis de Contribución}

\begin{table}[h]
\centering
\caption{Ingresos Promedio Anuales por Cultivo (5 años)}
\begin{tabular}{@{}lrrrr@{}}
\toprule
\textbf{Producto} & \textbf{Volumen} & \textbf{Precio} & \textbf{Ingreso} & \textbf{Contribución} \\
 & \textbf{(t/año)} & \textbf{(MXN/t)} & \textbf{(MXN)} & \textbf{(\%)} \\
\midrule
Frijol Jamapa & 86.0 & 35,000 & 3,008,544 & 43.2\% \\
Pepita calabaza & 26.4 & 80,000 & 2,114,112 & 30.4\% \\
Maíz forrajero & 230.0 & 8,000 & 1,839,974 & 26.4\% \\
\midrule
\rowcolor{tmlight}
\textbf{TOTAL} & --- & --- & \textbf{6,962,630} & \textbf{100\%} \\
\bottomrule
\end{tabular}
\end{table}

\subsection{Gráfica de Composición de Ingresos}

\begin{center}
\begin{tikzpicture}
\begin{axis}[
    ybar,
    bar width=1.2cm,
    width=\textwidth,
    height=8cm,
    ylabel={Ingresos anuales (MXN)},
    xlabel={Producto},
    xtick=data,
    xticklabels={Frijol Jamapa, Pepita Calabaza, Maíz Forrajero},
    ymin=0,
    ymax=3500000,
    nodes near coords,
    nodes near coords align={vertical},
    every node near coord/.append style={font=\small, /pgf/number format/.cd, fixed, precision=0, set thousands separator={,}},
    ymajorgrids=true,
    grid style=dashed,
]
\addplot[fill=tmgreen] coordinates {(1,3008544) (2,2114112) (3,1839974)};
\end{axis}
\end{tikzpicture}
\end{center}

\subsection{Estrategia de Mercado}

\begin{itemize}
  \item \textbf{Frijol Jamapa (43\% ingresos):} Variedad premium yucateca con alta demanda nacional. Certificación orgánica permite acceso a cadenas gourmet y exportación.
  
  \item \textbf{Pepita calabaza (30\% ingresos):} Mercado especializado (snacks saludables, panadería artesanal). Pepita orgánica mexicana tiene prestigio internacional.
  
  \item \textbf{Maíz forrajero (26\% ingresos):} Elimina dependencia de proveedores externos para granjas avícolas. Valor calculado como costo evitado (ahorro de \$1.84M anuales en compras).
\end{itemize}

\section{Análisis de Sensibilidad}

\subsection{Escenarios de Precio}

\begin{table}[h]
\centering
\caption{Impacto de Variaciones de Precio en ROI 5 Años}
\begin{tabular}{@{}lrrr@{}}
\toprule
\textbf{Escenario} & \textbf{Variación} & \textbf{Ganancia Neta} & \textbf{ROI} \\
 & \textbf{Precios} & \textbf{5 años (MXN)} & \textbf{(\%)} \\
\midrule
Pesimista & $-20\%$ & 18,183,152 & 380.3\% \\
\rowcolor{tmlight}
Base & 0\% & 25,933,152 & 542.5\% \\
Optimista & $+20\%$ & 33,683,152 & 704.7\% \\
\bottomrule
\end{tabular}
\end{table}

\textbf{Observación:} Incluso con caída de 20\% en precios orgánicos, el proyecto mantiene ROI $>$380\%, demostrando robustez financiera.

\subsection{Escenarios de Productividad}

\begin{table}[h]
\centering
\caption{Impacto de Variaciones en Rendimiento}
\begin{tabular}{@{}lrrr@{}}
\toprule
\textbf{Escenario} & \textbf{Rendimiento} & \textbf{Ganancia Neta} & \textbf{ROI} \\
 & \textbf{vs Base} & \textbf{5 años (MXN)} & \textbf{(\%)} \\
\midrule
Bajo (sequía/plagas) & $-15\%$ & 19,858,152 & 415.4\% \\
\rowcolor{tmlight}
Base (SPCM validado) & 0\% & 25,933,152 & 542.5\% \\
Alto (condiciones óptimas) & $+15\%$ & 32,008,152 & 669.6\% \\
\bottomrule
\end{tabular}
\end{table}

\subsection{Riesgos Identificados y Mitigación}

\begin{table}[h]
\centering
\small
\caption{Matriz de Riesgos}
\begin{tabular}{@{}p{3cm}p{4cm}p{5cm}@{}}
\toprule
\textbf{Riesgo} & \textbf{Impacto Potencial} & \textbf{Estrategia de Mitigación} \\
\midrule
Huracanes & Pérdida 1 ciclo ($-$33\% año) & Escalonamiento de siembras, tolchés protectoras \\
\midrule
Plagas & $-$10-20\% rendimiento & Control biológico con aves de pastoreo, diversidad policultivo \\
\midrule
Sequía extraordinaria & Falla de ciclo & Pozos profundos con reserva 6 meses, mulching \\
\midrule
Caída precios orgánicos & $-$15-25\% ingresos & Integración vertical (autoconsumo avícola), contratos anticipados \\
\midrule
Mano de obra & Incremento costos 30\% & Capacitación comunidad local, mecanización gradual \\
\bottomrule
\end{tabular}
\end{table}

\newpage
\section{Proyección Evolutiva 5 Años}

\subsection{Gráfica de Ganancia Acumulada}

\begin{center}
\begin{tikzpicture}
\begin{axis}[
    width=\textwidth,
    height=10cm,
    xlabel={Año},
    ylabel={Ganancia Acumulada (MXN)},
    xmin=0, xmax=5,
    ymin=-5000000, ymax=30000000,
    xtick={0,1,2,3,4,5},
    ytick={-5000000,0,5000000,10000000,15000000,20000000,25000000,30000000},
    yticklabel style={/pgf/number format/.cd, fixed, precision=0, set thousands separator={,}},
    legend pos=north west,
    ymajorgrids=true,
    xmajorgrids=true,
    grid style=dashed,
]

% Línea de inversión inicial
\addplot[color=red, very thick, dashed] coordinates {(0,-4780000) (5,-4780000)};
\addlegendentry{Inversión inicial}

% Línea de punto de equilibrio
\addplot[color=black, thick, dotted] coordinates {(0,0) (5,0)};
\addlegendentry{Punto de equilibrio}

% Ganancia acumulada
\addplot[color=tmgreen, very thick, mark=*, mark size=3pt] coordinates {
    (0,-4780000)
    (1,729664)
    (2,6872294)
    (3,13014924)
    (4,19474038)
    (5,25933152)
};
\addlegendentry{Ganancia neta acumulada}

\end{axis}
\end{tikzpicture}
\end{center}

\textbf{Análisis:} La inversión inicial de \$12.43M se recupera completamente al final del \textbf{Año 2}, con flujo de caja positivo sostenido a partir del tercer año. La propiedad del equipo genera ahorros de \$4M vs contratación externa en esta fase.

\subsection{Evolución de Productividad}

\begin{center}
\begin{tikzpicture}
\begin{axis}[
    width=\textwidth,
    height=8cm,
    xlabel={Año},
    ylabel={Producción Total (toneladas)},
    xmin=1, xmax=5,
    ymin=0, ymax=400,
    xtick={1,2,3,4,5},
    legend pos=north west,
    ymajorgrids=true,
    grid style=dashed,
    ybar stacked,
    bar width=15pt,
]

% Maíz
\addplot[fill=tmgreen!70] coordinates {(1,209.1) (2,230.0) (3,230.0) (4,240.5) (5,240.5)};
\addlegendentry{Maíz}

% Frijol
\addplot[fill=tmbrown!70] coordinates {(1,78.1) (2,86.0) (3,86.0) (4,89.9) (5,89.9)};
\addlegendentry{Frijol}

% Pepita
\addplot[fill=orange!70] coordinates {(1,24.0) (2,26.4) (3,26.4) (4,27.6) (5,27.6)};
\addlegendentry{Pepita}

\end{axis}
\end{tikzpicture}
\end{center}

\newpage
\section{Conclusiones y Recomendaciones}

\subsection{Viabilidad Financiera}

El análisis demuestra \textbf{viabilidad económica excepcional} del sistema de milpa tecnificada para Terra Maya Orgánica:

\begin{enumerate}
  \item \textbf{Recuperación completa:} Inversión inicial recuperada al finalizar Año 2
  \item \textbf{Rentabilidad sostenida:} ROI 147\% a 5 años (promedio 29.4\%/año)
  \item \textbf{Ventaja estratégica:} Equipo propio ahorra \$16.9M en expansión a 250 ha
  \item \textbf{Robustez financiera:} Proyecto viable incluso con caídas de 20\% en precios o rendimientos
  \item \textbf{Diversificación de ingresos:} Tres productos comerciales reducen riesgo de mercado
\end{enumerate}

\subsection{Beneficios Estratégicos}

Más allá del retorno financiero directo, el proyecto aporta:

\begin{itemize}
  \item \textbf{Autosuficiencia avícola optimizada:} Con sistema FVH, 20 ha (210 t/año maíz) alimentan 100,000 pollos/año consumiendo solo 70 t, liberando 140 t para comercialización
  \item \textbf{Sostenibilidad ambiental:} Cero roza-tumba-quema, captura de carbono, conservación de selva
  \item \textbf{Impacto social:} Generación de empleo para mujeres rurales mayas (25 empleos directos/20 ha + sala FVH)
  \item \textbf{Innovación científica:} Validación de SPCM en producción orgánica comercial + integración FVH inédita en Yucatán
  \item \textbf{Fortalecimiento de marca:} Integración vertical orgánica certificada (única en región) con diferenciación "pollos alimentados con maíz germinado maya"
  \item \textbf{Eficiencia máxima de recursos:} Sistema FVH reduce demanda de maíz 68\%, permitiendo escalamiento avícola sin expandir milpa proporcionalmente
\end{itemize}

\subsubsection{Impacto del Sistema FVH en Escalamiento}

\begin{tcolorbox}[colback=green!5!white, colframe=green!75!black, title=Liberación de Hectáreas Comerciales (Escenario 250 Ha)]
\textbf{Modelo convencional (sin FVH):}
\begin{itemize}
  \item 500,000 pollos/año requieren 1,100 t maíz = 105 ha dedicadas
  \item Quedan 145 ha para comercialización pura
  \item Ingreso comercial: 145 ha por \$691,800/ha = \textbf{\$100.3M/año}
\end{itemize}

\textbf{Modelo FVH (maíz germinado):}
\begin{itemize}
  \item 500,000 pollos/año requieren solo 350 t maíz = 33 ha dedicadas
  \item Quedan \textbf{217 ha para comercialización pura}
  \item Ingreso comercial: 217 ha por \$691,800/ha = \textbf{\$150.1M/año}
  \item \textbf{Ganancia incremental: \$49.8M/año (+50\% ingresos)}
\end{itemize}

\textbf{Inversión FVH (sala 500k pollos/año):}
\begin{itemize}
  \item Infraestructura: \$1.45M (1,200 m² con charolas automatizadas)
  \item Ahorro maíz vs convencional: 750 toneladas anuales por \$8,000/tonelada = \$6.0M/año
  \item Valor tierras liberadas: \$49.8M/año
  \item \textbf{ROI total: 3,848\%} (payback 31 días)
\end{itemize}
\end{tcolorbox}

\subsection{Recomendaciones de Implementación}

\subsubsection{Fase 1: Lote Piloto (Año 1)}
\begin{itemize}
  \item Implementar 20 ha validando todos los supuestos del modelo
  \item Establecer biofábricas in-situ para producción de biofertilizantes
  \item Certificar orgánico desde ciclo 1 (transición 0 por suelo virgen)
  \item Documentar protocolos para escalamiento
\end{itemize}

\subsubsection{Fase 2: Escalamiento (Años 2-3)}
\begin{itemize}
  \item Expandir a 100 ha con capital generado y equipo amortizado (Años 3-5)
  \item Mecanizar excavación de pocetas (reducir costo 20\%)
  \item Establecer contratos anticipados con distribuidores orgánicos
  \item Integrar procesamiento de pepita (descascarado in-situ, +40\% valor agregado)
\end{itemize}

\subsubsection{Fase 3: Consolidación (Años 4-5)}
\begin{itemize}
  \item Alcanzar 250 ha objetivo
  \item Exportación directa de frijol Jamapa a mercados internacionales
  \item Turismo agroecológico (educación milpa orgánica)
  \item Replicación del modelo en comunidades mayas asociadas
\end{itemize}

\subsection{Factores Críticos de Éxito}

\begin{enumerate}
  \item \textbf{Certificación orgánica:} Mantener estándares para precios premium (43-80\% sobre convencional)
  \item \textbf{Gestión hídrica:} Garantizar operación de 3 ciclos/año con riego confiable
  \item \textbf{Control biológico:} Integración aves-cultivos para reducir plagas sin químicos
  \item \textbf{Calidad de semilla:} Selección masal criollo adaptado a condiciones locales
  \item \textbf{Capacitación continua:} Personal técnico especializado en SPCM y agricultura orgánica
\end{enumerate}

\newpage
\section*{Anexos}

\subsection*{A. Metodología de Cálculo}

\textbf{Rendimiento por cultivo:}
\begin{align*}
\text{Plantas/ha} &= \text{Pocetas/ha} \times \text{Semillas/poceta} \times \text{Supervivencia} \\
\text{Rend. anual (t/ha)} &= \text{Plantas/ha} \times \text{Rend./planta/ciclo} \times 3 \text{ ciclos}
\end{align*}

\textbf{Ingresos anuales:}
\begin{align*}
\text{Ingresos}_{\text{año}} &= \sum_{i=1}^{3} (\text{Producción}_i \times \text{Precio}_i \times \text{Factor mejora})
\end{align*}

\textbf{ROI acumulado:}
\begin{align*}
\text{ROI}_{\text{acum}} &= \frac{\sum \text{Ganancias netas} - \text{Inversión inicial}}{\text{Inversión inicial}} \times 100\%
\end{align*}

\subsection*{B. Referencias}
\addcontentsline{toc}{subsection}{Referencias}

\begin{thebibliography}{99}

\bibitem{cicy2018}
Larqué Saavedra, A., Nexticapan Garcéz, Á., \& Caamal Maldonado, A. (2018). Sistema de producción continua de maíz en Yucatán. Centro de Investigación Científica de Yucatán (CICY).

\bibitem{fira2024}
Fideicomisos Instituidos en Relación con la Agricultura. (2024). \textit{Costos de referencia: Sistemas de riego tecnificado}. Banco de México.

\bibitem{siap2025}
Servicio de Información Agroalimentaria y Pesquera. (2025). \textit{Anuario estadístico de la producción agrícola}. SAGARPA.

\end{thebibliography}

\subsection*{C. Datos de Contacto}

\begin{tabular}{ll}
\textbf{Empresa:} & Terra Maya Orgánica \\
\textbf{Representante:} & Lic. Carlos Sobrino Sierra \\
\textbf{Sitio web:} & https://terramaya.mx/ \\
\textbf{Ubicación:} & Timucuy, Yucatán, México \\
\textbf{Certificaciones:} & Orgánico nacional e internacional \\
\end{tabular}

\vfill

\begin{center}
\textcolor{tmgreen}{\rule{\textwidth}{2pt}}

\large\textbf{Terra Maya Orgánica}

\textit{Agricultura orgánica maya con tecnología científica}

\textit{Produciendo alimentos sanos mientras conservamos nuestra selva}

\textcolor{tmgreen}{\rule{\textwidth}{2pt}}
\end{center}

\end{document}
